\section{Bilder}
\shorthandoff{"}
\subsection{Bilddateien}

\begin{frame}[fragile]
\frametitle{Dateien einbinden}
\begin{itemize}[<+->]
  \item um externe Bilder in \LaTeX\ einzubetten muss in der Präambel das Paket \texttt{graphicx} durch \\
\begin{center}
\lstinline[style=Latex]+\usepackage{graphicx}+
\end{center}
eingebunden werden \vfill
  \item ein Bild wird mittels \lstinline[style=Latex]+\includegraphics[<Parameter>]{<Datei>}+ eingebunden. \\
Ein wichtiger Parameter ist
  \begin{itemize}
 %   \item \texttt{angle}: dreht das Bild um einen bestimmten Winkel
    \item \texttt{scale}: streckt das Bild um einen Faktor (beachte \lstinline[style=Latex]+\textwidth+)
 %   \item \texttt{height/width}: legt die Höhe/Breite fest
 %   \item \texttt{bb}: definiert die Grenzen der Bilddatei in vier Werten
 %   \item \texttt{clip}: wenn auf \texttt{true} gesetzt und zusammen mit \texttt{bb} verwendet, wird das Bild entsprechend zugeschnitten
  \end{itemize}
 % \item mit \lstinline[style=Latex]+\graphicspath{<Pfad>}+ kann angegeben werden, wo die Bilder liegen
 % \item über \lstinline[style=Latex]+\DeclareGraphicsExtensions{<Endungen>}+ kann definiert werden, nach welchen Bildern standardmäßig gesucht wird
\end{itemize}
\end{frame}

\begin{frame}[fragile]
\frametitle{figure-Umgebung}
\begin{itemize}[<+->]
  \item die \texttt{figure}-Umgebung ist eine Gleitbox für Bilder
  \item die Syntax ist
  \begin{lstlisting}[style=Latex]
\begin{figure}[<Pos>]
<Bild>
\end{figure}  
  \end{lstlisting}
  \item die Position wird angegeben durch verschiedene Buchstaben
  \begin{itemize}
    \item[h] das Bild wird auf der Momentanen Seite eingebunden
    \item[t/b] das Bild wird oben / unten auf der Seite eingebunden
    \item[p] das Bild kommt auf eine eigene Seite
  \end{itemize}
  \item zusätzlich ist der \lstinline[style=Latex]+\caption{<Text>}+-Befehl verfügbar, der dem Bild einen Untertitel verschafft
 % \item viele Bilder können mittels \lstinline[style=Latex]+\subfigure+ nebeneinander eingefügt werden
\end{itemize}
\end{frame}

   
\begin{frame}[fragile]
\frametitle{Beispiel}
\begin{lstlisting}[style=Latex]
\begin{figure}[h]
  \includegraphics[width=.4\textwidth]{hypnotoad.png}
  \caption{all glory to the hypnotoad}
\end{figure}
\end{lstlisting} \vspace{-25pt}
\pause
\begin{figure}[h]
  \includegraphics[width=.4\textwidth]{images/hypnotoad.png}
  \caption{all glory to the hypnotoad}
\end{figure}
\end{frame}

\begin{comment}
\subsection{TikZ}
\begin{frame}
\frametitle{TikZ}
\begin{itemize}[<+->]
  \item \textbf{T}ikZ \textbf{i}st \textbf{k}ein \textbf{Z}eichenprogramm
  \item Vektorgrafiken direkt in \LaTeX\, erstellen
  \item Vorteile:
  \begin{itemize}
    \item Alle Vorteile von \LaTeX
    \item fügt sich nahtlos ein
    \item ununterbrochener Workflow
    \item schnell
  \end{itemize}
  \item Nachteile:
  \begin{itemize}
    \item Ergebnis erst zum Schluss
    \item Einarbeitungszeit
  \end{itemize}
\end{itemize}
\end{frame}

\begin{frame}[fragile]
\frametitle{TikZ}
\begin{itemize}[<+->]
  \item TikZ wird mittels \lstinline[style=Latex]+\usepackage{tikz}+ geladen
  \item über \lstinline[style=Latex]+\usetikzlibrary{<library>}+ können zusätzliche Module geladen werden
  \item Nutzung über \lstinline[style=Latex]+\tikz{<Befehle>}+ oder die \texttt{tikzpicture}-Umgebung.
  \item Syntax: \lstinline[style=Latex]+\<Befehl>[<Parameter>](<Koordinate>){<Objekte>};+ 
  \item bei Verschachtelung mehrerer Befehle erhält nur der erste Befehl einen \lstinline[style=Latex]+\+: \\
  	\lstinline[style=Latex]+\draw (0,0) -- coordinate (b) (1,1);+  	
\end{itemize}
\end{frame}

\begin{frame}[fragile]
\frametitle{Koordinaten}
\begin{itemize}[<+->]
  \item Koordinaten können auf verschiedene Weisen angegeben werden:
  \begin{itemize}
    \item kartesisch: (X,Y)
    \item polar: ($\alpha$:r)
    \item Name: (Name)
  \end{itemize}
  \item Gültige Einheiten sind alle \TeX-Einheiten, Standard ist cm.
  \item Koordinaten können mittels 
  \begin{lstlisting}[style=Latex]
\coordinate (<Name>) at (<Position>);
  \end{lstlisting}
  gespeichert werden.
  \begin{itemize}
    \item steht \lstinline[style=Latex]+\coordinate+ direkt hinter einer Koordinate, so wird diese als Position übernommen
    \item steht \lstinline[style=Latex]+\coordinate+ direkt hinter einer Verbindungslinie (\lstinline[style=Latex]+--+), so wird die Mitte der Strecke zwischen den beiden Koordinaten übergeben
  \end{itemize}
\end{itemize}
\end{frame}

\begin{frame}[fragile]
\frametitle{Berechnungen}
\begin{itemize}[<+->]
  \item Nach dem Einbinden des Pakets \texttt{calc} kann man mit Koordinaten rechnen:
  \begin{itemize}
    \item Punkt auf einer Geraden: (\$(K1)!.3!(K2)\$)
    \item Addition: (\$(K1)+(K2)\$)
    \item Addition zur letzten: ++(K1)
  \end{itemize}
  \item noch komplexere Berechnungen funktionieren mittels \texttt{intersection}:\\
     \begin{lstlisting}[style=Latex]
(intersection cs: first line={(0,0)--(2,1)}, 
  second line={(0,1)--(2,-1)})
\end{lstlisting}
     berechnet den Schnittpunkt der Geraden durch (0,0)-(2,1) und (0,1)-(2,-1).
     \item TikZ kann noch viel mehr, mehr dazu in der Dokumentation
\end{itemize}
\end{frame}

\begin{frame}[fragile]
\frametitle{Beispiel}
\begin{lstlisting}[style=Latex]
\begin{tikzpicture}
\coordinate (a) at (0,0);
\coordinate (b) at (3,3);
\draw (a) -- (b) 
  -- ++ (-1,-2) coordinate (c) 
  -- ($(a)!.5!(b)$) coordinate (d);
\draw (0,2) -- (intersection cs: 
  first line={(a)--(b)}, second line={(c)--(d)});
\end{tikzpicture}
\end{lstlisting}
\begin{center}
  \begin{tikzpicture}[thick,pkt/.style={circle,fill=red,inner sep = 3pt}]
    \coordinate (a) at (0,0); 
    \coordinate (b) at (3,3); 
    \draw[thin,gray,step=1] (0,0) grid (3,3);
    \draw (a) -- (b) -- ++ (-1,-2) coordinate (c) 
    -- ($(a)!.5!(b)$) coordinate (d);
    \pause 
    \draw (0,2) -- (intersection cs: 
    first line={(a)--(b)}, second line={(c)--(d)});
  \end{tikzpicture}
\end{center}
\end{frame}

\begin{frame}[fragile]
\frametitle{Pfeile}
\newcommand\entry[1]{\tikz{\draw[#1] (0,0) -- ++(1,0);}}
\begin{itemize}
  \item<1-> der Befehl zum ändern des Endpfeils ist \lstinline[style=Latex]+*linke Spitze*-*rechte Spitze*+
  \item<2-> Spitzentypen sind
  \begin{center}\begin{tabular}{l|l||l|l}
    Code 		& Aussehen 		& Code		& Aussehen \\\hline
    \lstinline[style=Latex]+-+ 	& \entry{-} 		& \lstinline[style=Latex]+-o+ 	& \entry{-o}\\
    \lstinline[style=Latex]+->+ 	& \entry{->} 	& \lstinline[style=Latex]+-<+ 	& \entry{-<}\\
    \lstinline[style=Latex]+->>+ 	& \entry{->>} 	& \lstinline[style=Latex]+-[+ 	& \entry{-[}\\
    \lstinline[style=Latex]+->|+ 	& \entry{->|} 	& \lstinline[style=Latex]+-)+ 	& \entry{-)}\\
    \lstinline[style=Latex]+-*+ 	& \entry{-*} 	& \lstinline[style=Latex]+-(+ 	& \entry{-(}
  \end{tabular}\end{center} \pause
  \item<3-> Beispiel:
    \begin{lstlisting}[style=Latex]
\draw[*->>] (0,0) -- (3,0.5) -- (6,0);
    \end{lstlisting}
    \tikz{\draw[*->>] (0,0) -- (3,0.5) -- (6,0);}
  \item<4-> Das Paket \texttt{arrows} stellt noch viel mehr Pfeile zur Verfügung
\end{itemize}
\end{frame}

\begin{frame}[fragile]
\frametitle{Stricharten}
\newcommand\entry[1]{\texttt{#1}&\tikz{\draw[#1] (0,0) -- ++(1,0);}}
\begin{itemize}[<+->]
  \item Strichstärke: \\
  \begin{center}\begin{tabular}{l|l||l|l}
    Code 		& Aussehen 	& Code 	& Aussehen 	\\\hline
    \entry{ultra thin} & \entry{very thin} \\
    \entry{thin} & \entry{semithick} \\
    \entry{thick} & \entry{very thick} \\
    \entry{ultra thick}
  \end{tabular}\end{center}
  \item Strichart:\\[2mm]
  \hspace*{-8mm}\begin{tabular}{l|l||l|l}
    Code 		& Aussehen 	& Code 	& Aussehen 	\\\hline
    \entry{solid} & \entry{dotted} \\
    \entry{loosely dotted} & \entry{dashed} \\
    \entry{densely dashed} & \entry{loosely dashed} 
  \end{tabular}
\end{itemize}
\end{frame}


\begin{frame}[fragile]
\frametitle{weitere Opt-Argumente}
\begin{itemize}[<+->]
  \item \texttt{rounded corners}: rundet die Ecken ab.
  \item \texttt{bend left / bend right}: erzeugt eine Kurve statt einer geraden Linie. Muss mit \texttt{to} statt \lstinline[style=Latex]+--+ verwendet werden.
  \item \texttt{fill}: füllt die Zeichnung. Mit ``\texttt{=}'' kann eine Farbe definiert werden \\
    Mit \lstinline[style=Latex]+fill=orange+ wird der Bereich orange gefärbt.
  \item \texttt{draw}: färbt die Linie. Mit ``\texttt{=}'' kann eine Farbe definiert werden \\
    Mit \lstinline[style=Latex]+draw=blue+ wird die Linie blau gefärbt.
  \item \texttt{line width}: Legt mittels ``\texttt{=}'' die Strichstärke fest.\\
    Mit \lstinline[style=Latex]+line width=4mm+ wird die Linie 4mm dick.
  \item \texttt{dash pattern:} Erzeugt eine Strichart. Beispiel:\\
    \lstinline[style=Latex]+dash pattern=on 2pt off 3pt on 4pt off 4pt+:\\
    \tikz{\draw[dash pattern=on 2pt off 3pt on 4pt off 4pt] (0,0) -- (8,0);}
\end{itemize}
\end{frame}

\begin{frame}[fragile]
\frametitle{Beispiel}
\begin{lstlisting}[style=Latex]
\begin{tikzpicture}
\draw[dashed,->] (0,0) -- (1,1);
\draw[fill=green,draw=blue,rounded corners] 
  (2,0) -- ++(1,0) -- ++(0,1) -- ++(-1,0) 
  -- cycle;
\draw[thick,bend left] (4,0) to (5,1);
\draw[blue] (6,0) -| (7,1) -- (8,1) |- (9,0);
\end{tikzpicture}
\end{lstlisting}
\vfill
\pause
\begin{center}
  \begin{tikzpicture}
    \draw[dashed,->] (0,0) -- (1,1); \pause
    \draw[fill=green,draw=blue,rounded corners] (2,0) -- ++(1,0) --
    ++(0,1) -- ++(-1,0) -- cycle; \pause \draw[thick,bend left] (4,0)
    to (5,1); \pause \draw[blue] (6,0) -| (7,1) -- (8,1) |- (9,0);
  \end{tikzpicture}
\end{center}
\end{frame}

\begin{frame}[fragile]
\frametitle{Figuren}
\begin{itemize}[<+->]
  \item Rechteck: \lstinline[style=Latex]+\draw (0,0) rectangle (1,2);+ \tikz[scale=0.5,baseline={([yshift=-.8ex]current bounding box.center)}]{\draw (0,0) rectangle (2,1);}
  \item Kreis: \lstinline[style=Latex]+\draw (0,0) circle (1);+ \tikz[scale=0.5,baseline={([yshift=-.8ex]current bounding box.center)}]{\draw (0,0) circle (1);}
  \item Ellipse: \lstinline[style=Latex]+\draw (0,0) ellipse (2 and 1);+ \tikz[scale=0.5,baseline={([yshift=-.8ex]current bounding box.center)}]{\draw (0,0) ellipse (2 and 1);}
  \item Bogen: \lstinline[style=Latex]+\draw (0,0) arc (0:45:1cm);+ \tikz[baseline={([yshift=-.8ex]current bounding box.center)}]{\draw (0,0) arc (0:45:1cm);}
  \item Gitter: \lstinline[style=Latex]+\draw[step=0.2] (0,0) grid (2,1);+ \tikz[scale=0.5,baseline={([yshift=-.8ex]current bounding box.center)}]{\draw[step=0.2] (0,0) grid (2,1);} 
\end{itemize}
\end{frame}

\begin{frame}[fragile]
\newcommand\entry[1]{\tikz[baseline={([yshift=-.8ex]current bounding box.center)}]{\node[draw,#1] {Test};}}
\frametitle{Knoten}
\begin{itemize}[<+->]
  \item Knoten dienen Dazu, ganzen Objekten eine Koordinate zu geben.
  \item Syntax: \lstinline[style=Latex]+\node[<Opt>](<Name>)at(<Koordinate>){<Objekt>};+
  \item steht der \lstinline[style=Latex]+\node+-Befehl hinter einer Koordinate, so wird diese übernommen.
  \item Standard-Knotenrahmen (jeweils mit der Option \texttt{draw}):
  \begin{center}\begin{tabular}{l|c|c|c}
  Opt-Parameter & \texttt{rectangle} & \texttt{circle} & \texttt{diamond} \\\hline
  Aussehen & \entry{rectangle} & \entry{circle} & \entry{diamond} 
  \end{tabular}\end{center}
  \item das Paket \texttt{shapes} stellt noch weitere zur Verfügung
\end{itemize}
\end{frame}

\begin{frame}[fragile]
\frametitle{Zusatzmöglichkeiten durch Knoten}
\begin{itemize}[<+->]
  \item steht der \lstinline[style=Latex]+\node+-Befehl hinter einem Verbindungsstrich zwischen zwei Koordinaten, so wird die Mitte der Strecke als Koordinate übergeben.
  \item zusätzliche Opt-Parameter:
  \begin{itemize}
    \item \texttt{atop/below/right/left}: verschiebt den Knoten relativ zur letzten Position
    \item \texttt{label}: erzeugt eine Beschriftung. Syntax:\\
    \lstinline[style=Latex]+\node[label=90:Text hier] {};+
  \end{itemize}
  \item Hat man dem Knoten einen Namen gegeben, so kann man mit \lstinline[style=Latex]+(<Name>.<Grad>)+ die Koordinaten am Rand des Objekts nutzen.\\
    Außerdem sind entsprechend die Gradzahlen \texttt{.north}, \texttt{.south}, \texttt{.east} und \texttt{.west} definiert.
\end{itemize}
\end{frame}


\begin{frame}[fragile]
\frametitle{Beispiel}
\begin{lstlisting}[style=Latex]
\begin{tikzpicture}
\draw (0,4) -- node[above] {Hallo Welt} (6,4);
\node[draw,diamond] (N1) at (0,1) {$N_1$};
\node[draw,circle,label=45:$\Phi$] (N2) at (4,1) {$N_2$};
\draw[->] (N1) -- (N2);
\end{tikzpicture}
\end{lstlisting}
\begin{center}\begin{tikzpicture}
\draw (0,4) -- node[above] {Hallo Welt} (6,4);
\end{tikzpicture}\end{center}
\end{frame}

\begin{frame}[fragile]
\frametitle{Beispiel}
\begin{lstlisting}[style=Latex]
\begin{tikzpicture}
\draw (0,4) -- node[above] {Hallo Welt} (6,4);
\node[draw,diamond] (N1) at (0,1) {$N_1$};
\node[draw,circle,label=45:$\Phi$] (N2) at (4,1) {$N_2$};
\draw[->] (N1) -- (N2);
\end{tikzpicture}
\end{lstlisting}
\begin{center}\begin{tikzpicture}
\draw (0,4) -- node[above] {Hallo Welt} (6,4);
\node[draw,diamond] (N1) at (1,3) {$N_1$};
\node[draw,circle,label=45:$\Phi$] (N2) at (5,3) {$N_2$};
\draw[->] (N1) -- (N2);
\end{tikzpicture}\end{center}
\end{frame}

\begin{frame}[fragile]
\frametitle{Weiterführendes}
\begin{itemize}[<+->]
  \item Argumente, die der Umgebung übergeben werden, gelten für die gesamte Zeichnung
  \item per \texttt{<Name>/.style={<Argumente>}} können eigene Stile definiert werden, die danach als Paket übergeben werden.
  \item per \texttt{<Position> = of <Name>} können Knoten relativ zu einem anderen Knoten gesetzt werden. Die Entfernung wird über die Variable \texttt{node distance} bestimmt. Als \texttt{<Position>} sind gültig \texttt{below}, \texttt{above}, \texttt{left}, \texttt{right}.
  \item per \texttt{shorten >= <Entfernung>} bzw. \texttt{shorten <= <Entfernung>} werden Linien rechts bzw. links um \texttt{<Entfernung>} gekürzt. Dies ist praktisch, um die Pfeilspitzen besser sichtbar zu machen.
  \item Farben können mit \texttt{<Farbe1>!<Prozent>!<Farbe2>} zusammengemischt werden. Beispielsweise ergibt \texttt{black!60!white} ein 60\%-iges Grau.
  \item mit \texttt{opacity=<Faktor>} kann ein Objekt transparent gemacht werden
  \item mit \texttt{scale=<Faktor>} werden in einem Objekt (oder für die gesamte Zeichnung) die Maßeinheiten gestreckt / gestaucht. Text bleibt unverändert.
\end{itemize}
\end{frame}

\begin{frame}
\frametitle{Bäume}
\begin{itemize}[<+->]
\item Mit Hilfe der \texttt{child}-Operation können Bäume erzeugt werden
\item dabei versucht jeder Knoten seine ``Kinder'' gleichmäßig unter sich zu verteilen
\item der Abstand auf jeder Ebene kann mit dem Argument \texttt{sibling distance} gesetzt werden
\item für jede Ebene kann durch Verändern des \texttt{level x} Stiles das Aussehen verändert werden
\item das Argument \texttt{missing} lässt Platz für einen nicht vorhandenen Knoten
\end{itemize}
\end{frame}

\begin{frame}[fragile]
  \frametitle{Beispiel}
\begin{columns}
\column{.6\textwidth}
  \begin{lstlisting}[style=Latex]
\begin{tikzpicture}
[every node/.style={circle,draw}, 
 level 1/.style={sibling distance=20mm},
 level 2/.style={sibling distance=10mm}]
\node {3}
  child {node {4} 
    child {node {5}}
    child {node {8}}
  }
  child {node {7}
    child[missing]
    child {node {11}}
  };
\end{tikzpicture}
  \end{lstlisting}
\pause
\column{.4\textwidth}
\begin{center}
\begin{tikzpicture}
[every node/.style={circle,draw}, 
 level 1/.style={sibling distance=20mm},
 level 2/.style={sibling distance=10mm}]
\node {3}
  child {node {4} 
    child {node {5}}
    child {node {8}}
  }
  child {node {7}
    child[missing]
    child {node {11}}
  };
\end{tikzpicture}
\end{center}
\end{columns}
\end{frame}

\begin{frame}[fragile]
\frametitle{Beispiel}
\begin{lstlisting}[style=Latex]
\begin{tikzpicture}[->,dotted,recht/.style={rectangle,
                        draw=green,fill=blue!20!white}]
\node[recht] (a) at (0,0) {Hallo};
\node[recht,right = of a] (b) at (0,0) {Welt};
\draw[shorten >= 2pt] (a) -- (b);
\end{tikzpicture}
\end{lstlisting}
\pause
\scalebox{3}{
\begin{tikzpicture}[->,dotted,recht/.style={rectangle,
                        draw=green, fill=blue!20!white}]
\node[recht] (a) at (0,0) {Hallo};
\node[recht,right = of a] (b) at (0,0) {Welt};
\draw[shorten >= 2pt] (a) -- (b);
\end{tikzpicture}}
\end{frame}


\begin{frame}[fragile]
\frametitle{foreach}
\begin{itemize}[<+->]
  \item Mit \texttt{foreach} lassen sich Werte in einer Liste automatisch abarbeiten
  \item Syntax: \lstinline[style=Latex]+\foreach \<Variable> in {<Liste>} {<Befehle>};+
  \item Beispiel: \lstinline[style=Latex]+\foreach \x in {1,2,3} {(\x)};+: \foreach \x in {1,2,3} {(\x)}
  \item \texttt{foreach} kann Listen automatisch vervollständigen: \lstinline[style=Latex]+{1,...,8}+
  \item Wertepaare können wie folgt angegeben werden: \lstinline[style=Latex]+\foreach \x/\y in {0/0,1/3,4/4} {(\x,\y)};+
\end{itemize}
\end{frame}

\begin{frame}[fragile]
\frametitle{Beispiel}
\begin{lstlisting}[style=Latex]
\begin{tikzpicture}
\foreach \x in {0,1,2,3}
  \draw (\x,0) circle (0.2cm);
\draw (0,0)
  \foreach \x in {1,...,3} { -- (\x,1) -- (\x,0) };
\end{tikzpicture}
\end{lstlisting}
\pause
\begin{center}\begin{tikzpicture}
\foreach \x in {0,1,2,3} 
  \draw (\x,0) circle (0.2);
  \draw (0,0)
  \foreach \x in {1,...,3} { -- (\x,1) -- (\x,0) };
\end{tikzpicture}
\end{center}
\end{frame}

\begin{frame}[fragile]
  \frametitle{Balkendiagramme}
  \begin{lstlisting}[style=Latex]
\begin{tikzpicture}
  \draw[->] (0,0) -- (0,5);
  \draw[->] (0,0) -- (8,0);
  \foreach \x/\y/\text in {1/3/A,2/4/B,3/2/C,4/1/D}
    \draw[-*] (\x,0) node[below] {\text} 
         -- (\x,\y) node[above] {\y\EUR};
\end{tikzpicture}
  \end{lstlisting}
\pause
  \begin{tikzpicture}
    \draw[->] (0,0) -- (0,5);
    \draw[->] (0,0) -- (8,0);
    \foreach \x/\y/\text in {1/3/A,2/4/B,3/2/C,4/1/D}
    \draw[-*] (\x,0) node[below] {\text} -- (\x,\y) node[above] {\y\EUR};
\end{tikzpicture}
\end{frame}

\begin{frame}[fragile]
  \frametitle{Kommutative Diagramme}
  \[ \begin{tikzpicture}[baseline= (a).base]
    \node[scale=2] (a) at (0,0){
  \begin{tikzcd}
    F_t(x) 
      \ar[r,Rightarrow,"\mathcal{B}_t", red] 
      \ar[d,"\mathcal{B}_X"] 
     & F(x) \ar[d,"\mathcal{B}_T" blue] \\
    A_t
      \ar[green]{r}[red,below]{\mathcal{B}_T}[orange]{\exists}
      \ar[ru,dash, dashed]  
     & A 
  \end{tikzcd}};
  \end{tikzpicture}\]
\end{frame}

\begin{frame}[fragile]
  \frametitle{Kommutative Diagramme}
  \begin{itemize}[<+->]
  \item Aktivieren mittels \lstinline[style=Latex]+\usetikzlibrary{cd}+
  \item Verwenden in einer \texttt{tikz-cd}-Umgebung
  \item Die Einträge des Diagrams sind angeordnet wie eine Matrix:
    \begin{itemize}
    \item Spaltenumbruch mit \lstinline[style=Latex]+&+
    \item Zeilenumbruch mit \lstinline[style=Latex]+\\+
    \end{itemize}
  \end{itemize}
\end{frame}

\begin{frame}[fragile]
  \frametitle{Beispiel}
  \begin{lstlisting}[style=Latex]
    \[\begin{tikzcd}
      F_t(x) & F(x) \\
      A_t    & A 
    \end{tikzcd}\]
  \end{lstlisting}
  \pause
  \[\begin{tikzcd}
    F_t(x) & F(x) \\
    A_t    & A 
  \end{tikzcd}\]
\end{frame}

\begin{frame}[fragile]
  \frametitle{Pfeile}
  Syntax: \lstinline[style=Latex]+\ar[<Richtung>,<Parameter>]+
  \begin{itemize}[<+->]
  \item Als Richtungen können verwendet werden:
    \begin{itemize}
    \item \textbf{u}: hoch (up)
    \item \textbf{d}: runter (down)
    \item \textbf{l}: links (left)
    \item \textbf{r}: rechts (right)
    \end{itemize}
    \pause
    Auch Kombinationen sind möglich, z.B. \textbf{lld}
  \item Einige mögliche Parameter:
    \begin{itemize}
    \item Pfeilspitzen werden deskriptiv gewählt (z.B. Rightarrow, squiggly, hook, two heads, ...)
    \item Beschriftungen durch Anführungszeichen: \lstinline[style=Latex]+"text"+
    \item Farben als Farbname
    \item alle üblichen TikZ-Parameter
    \end{itemize}
  \end{itemize}
\end{frame}

\begin{frame}[fragile]
  \frametitle{Beispiel}
  \begin{lstlisting}[style=Latex]
    \[\begin{tikzcd}
      F_t(x) \ar[r] \ar[d] & F(x) \ar[d] \\
      A_t \ar[r] \ar[ru]   & A 
    \end{tikzcd}\]
  \end{lstlisting}
  \pause
  \[\begin{tikzcd}
    F_t(x) \ar[r] \ar[d] & F(x) \ar[d] \\
    A_t \ar[r] \ar[ru]   & A 
  \end{tikzcd}\]
\end{frame}

\begin{frame}[fragile]
  \frametitle{Beispiel}
  \begin{lstlisting}[style=Latex]
    \[\begin{tikzcd}
      F_t(x) \ar[r,Rightarrow] \ar[d] & F(x) \ar[d] \\
      A_t \ar[r] \ar[ru,dash, dashed]  & A 
    \end{tikzcd}\]
  \end{lstlisting}
  \[\begin{tikzcd}
    F_t(x) \ar[r,Rightarrow] \ar[d] & F(x) \ar[d] \\
    A_t \ar[r] \ar[ru,dash, dashed]  & A 
  \end{tikzcd}\]
\end{frame}

\begin{frame}[fragile]
  \frametitle{Beispiel}
  \begin{lstlisting}[style=Latex]
  \[\begin{tikzcd}
    F_t(x) 
      \ar[r,Rightarrow,"\mathcal{B}_t"] 
      \ar[d,"\mathcal{B}_X"] 
     & F(x) \ar[d,"\mathcal{B}_T"] \\
    A_t
      \ar[r,"\mathcal{B}_T" below, "\exists" above] 
      \ar[ru,dash, dashed]  
     & A 
  \end{tikzcd}\]
  \end{lstlisting}
  \[\begin{tikzcd}
    F_t(x) 
     \ar[r,Rightarrow,"\mathcal{B}_t"] 
     \ar[d,"\mathcal{B}_X"] 
     & F(x) \ar[d,"\mathcal{B}_T"] \\
    A_t
     \ar[r,"\mathcal{B}_T" below, "\exists" above] 
     \ar[ru,dash, dashed]  
     & A 
  \end{tikzcd}\]
\end{frame}

\begin{frame}[fragile]
  \frametitle{Beispiel}
  \begin{lstlisting}[style=Latex]
  \[\begin{tikzcd}
    F_t(x) 
      \ar[r,Rightarrow,"\mathcal{B}_t", red] 
      \ar[d,"\mathcal{B}_X"] 
     & F(x) \ar[d,"\mathcal{B}_T" blue] \\
    A_t
      \ar[green]{r}[red,below]{\mathcal{B}_T}[orange]{\exists}
      \ar[ru,dash, dashed]  
     & A 
  \end{tikzcd}\]
  \end{lstlisting}
  \[\begin{tikzcd}
    F_t(x) 
      \ar[r,Rightarrow,"\mathcal{B}_t", red] 
      \ar[d,"\mathcal{B}_X"] 
     & F(x) \ar[d,"\mathcal{B}_T" blue] \\
    A_t
      \ar[green]{r}[red,below]{\mathcal{B}_T}[orange]{\exists}
      \ar[ru,dash, dashed]  
     & A 
  \end{tikzcd}\]
\end{frame}

\begin{frame}
  \frametitle{PGF-Plots}
  \centering
  \begin{tikzpicture}
	\begin{polaraxis}
	\addplot+[mark=none,domain=0:720,samples=600] 
		{sin(2*x)*cos(2*x)}; 
	% equivalent to (x,{sin(..)cos(..)}), i.e.
	% the expression is the RADIUS
	\end{polaraxis}
\end{tikzpicture}
\end{frame}

\begin{frame}[fragile]
  \frametitle{PGF-Plots}
  \begin{itemize}[<+->]
  \item Benötigt das Paket \texttt{pgfplots}
  \item Kann zusätzliche Pakete laden mit \lstinline[style=Latex]+\usepgfplotslibrary+
  \item Einstellen der Version mittels \lstinline[style=Latex]+\pgfplotsset{compat=1.12}+
    \begin{itemize}
    \item Eventuell muss hier eine kleinere Version gewählt werden
    \end{itemize}
  \item Anleitung: \url{http://pgfplots.sourceforge.net/pgfplots.pdf}
  \end{itemize}
\end{frame}

\begin{frame}[fragile]
  \frametitle{Funktionen}
  \begin{lstlisting}[style=Latex]
    \begin{tikzpicture}
      \begin{axis}
        \addplot[ red, domain=-3e-3:3e-3, samples=201 ]{ exp(-x^2 /
          (2e-3^2)) / (1e-3 * sqrt(2*pi)) };
      \end{axis}
    \end{tikzpicture}
  \end{lstlisting}
  \pause
  \begin{center}
    \begin{tikzpicture}[scale=.8]
      \begin{axis}
        \addplot[ red, domain=-3e-3:3e-3, samples=201 ]{ exp(-x^2 /
          (2e-3^2)) / (1e-3 * sqrt(2*pi)) };
      \end{axis}
    \end{tikzpicture}
  \end{center}
\end{frame}

\begin{frame}[fragile]
  \frametitle{Kurven}
  \begin{columns}
    \column{.5\textwidth}
    \begin{lstlisting}[style=Latex]
\begin{tikzpicture}
  \begin{polaraxis}
  \addplot+[ 
    mark=none, 
    domain=0:720, 
    samples=600 
    ]
    {sin(2*x)*cos(2*x) };
  \end{polaraxis}
\end{tikzpicture}
    \end{lstlisting}
    \pause
    \column{.49\textwidth}
    \begin{tikzpicture}[scale=.8]
      \begin{polaraxis}
        \addplot+[mark=none,domain=0:720,samples=600]
        {sin(2*x)*cos(2*x)};
        % equivalent to (x,{sin(..)cos(..)}), i.e.
        % the expression is the RADIUS
      \end{polaraxis}
    \end{tikzpicture}
  \end{columns}
\end{frame}


\begin{frame}[fragile]
  \frametitle{Kurven}
  
  \begin{columns}
    \column{.5\textwidth}
    \begin{lstlisting}[style=Latex]
\begin{tikzpicture}
\begin{axis}[ 
  title={$x \exp(-x^2-y^2)$}, 
  xlabel=$x$, ylabel=$y$, 
  small]
  \addplot3[ 
    surf, 
    domain=-2:2, 
    domain y=-1.3:1.3,]
    {exp(-x^2-y^2)*x };
\end{axis}
\end{tikzpicture}
    \end{lstlisting}
    \pause
    \column{.49\textwidth}
    \begin{tikzpicture}
      \begin{axis}[title={$x \exp(-x^2-y^2)$},xlabel=$x$,
        ylabel=$y$,small,]
        \addplot3[surf,domain=-2:2,domain y=-1.3:1.3,]
        {exp(-x^2-y^2)*x};
      \end{axis}
  \end{tikzpicture}
  \end{columns}

\end{frame}


\begin{frame}
\frametitle{Dokumentation}
\url{www.ctan.org/get/graphics/pgf/base/doc/generic/pgf/pgfmanual.pdf}
\end{frame}
\end{comment}
