\section{Weiteres}

\subsection{selber definieren}

\begin{frame}[fragile]
\frametitle{Befehle definieren}
\begin{itemize}[<+->]
  \item mit \lstinline[style=Latex]+\newcommand{\<Befehl>}[<AnzArgs>][<Std>]{<wastun>}+ können eigene Befehle definiert werden
  \item Befehlsnamen dürfen nur aus Buchstaben bestehen
  \item mit \texttt{<AnzArgs>} kann die Anzahl der übergebenen Argumente festgelegt werden. Diese können dann mit \lstinline[style=Latex]+#1+, \lstinline[style=Latex]+#2+, \ldots abgerufen werden.
  \item mit \texttt{<Std>} kann ein Standardwert für das erste Argument gegeben werden. Dadurch wird dieses optional.
  \item \lstinline[style=Latex]+\renewcommand+ überschreibt einen bestehenden Befehl
  \item durch \lstinline[style=Latex]+\ensuremath+ wird der Befehl falls nötig in den Mathematikmodus versetzt
\end{itemize}
\end{frame}

\begin{frame}[fragile]
\frametitle{Beispiel -- Befehle}
\begin{lstlisting}[style=Latex]
\newcommand\eps{\ensuremath\varepsilon}
\newcommand\cboxes[2]{
  \begin{center}
    \begin{tabular}{|c|c|}\hline #1 & #2 \\\hline\end{tabular}
  \end{center}
}
\eps \\
\cboxes{Hallo}{Welt}
\end{lstlisting}
\newcommand\eps{\ensuremath\varepsilon}
\newcommand\cboxes[2]{
  \begin{center}
    \begin{tabular}{|c|c|}\hline #1 & #2 \\\hline\end{tabular}
  \end{center}
}
\eps \\
\cboxes{Hallo}{Welt}
\end{frame}



\subsection{Folien}

\begin{frame}[fragile]
\frametitle{Folien}
\begin{itemize}[<+->]
  \item als Dokumentklasse muss \texttt{beamer} verwendet werden
  \item es stehen viele verschiedene Themes zur Auswahl, einen Überblick erhält man unter \url{http://www.namsu.de/latex/themes/uebersicht_beamer.html} \\
    Diese werden mit \lstinline[style=Latex]+\usetheme{<Themenname>}+ aktiviert
  \item verwendet man \texttt{beamer} mit der Option \texttt{handout} werden alle Übergänge deaktiviert

\end{itemize}
\end{frame}

\begin{frame}[fragile]
\frametitle{frame-Umgebung}
\begin{itemize}[<+->]
  \item jede Folie ist in der Beamer-Klasse eine \texttt{frame}-Umgebung oder alternativ durch \lstinline[style=Latex]+\frame{<Inhalt>}+.
  \item mit \lstinline[style=Latex]+\frametitle{<Titel>}+ kann der Seitentitel festgelegt werden
  \item einen Untertitel erhält man mit \lstinline[style=Latex]+\subtitle{<Untertitel>}+
  \item eine vorgefertigte Titelseite erhält man mittels \lstinline[style=Latex]+\titlepage+
\end{itemize}
\end{frame}

\begin{frame}[fragile]
\frametitle{Overlays}
\begin{itemize}[<+->]
  \item der Befehl \lstinline[style=Latex]+\pause+ erzeugt eine Folie mit dem Inhalt bis zur Pause und eine weitere mit der gesamten Seite
  \item Aufzählungen erhalten als zusätzlichen optionalen Parameter \lstinline[style=Latex]+<<Pausenart>>+.\\
    Dabei bedeutet eine Zahl $n$, dass etwas in Folie $n$ passiert, ein ``+'' deckt das Item auf oder verdeckt es und ein ``-'' lässt es bestehen, ``alert@$n$'' markiert das Item in Folie $n$. Mit ``|'' können mehrere Anweisungen hintereinander stehen.
  \item steht eine Pausenart als optionaler Parameter an der Aufzählungsumgebung, so gilt dies für die gesamte Aufzählung inkl. Unteraufzählungen.
\end{itemize}
\end{frame}

\begin{frame}[fragile]
\frametitle{Beispiel}\vspace{-5pt}
\begin{lstlisting}[style=Latex]
\begin{itemize}
  \item Einleitung
  \item<2-> daher
  \item<3-|alert@3> aber Achtung!
  \item<4> ist nur in 4 da
  \item<5-> Schlussfolgerung
\end{itemize}
\end{lstlisting}\vspace{-25pt}\pause
\begin{itemize}
  \item Einleitung
  \item<2-> daher
  \item<3-|alert@3> aber Achtung!
  \item<4> ist nur in 4 da
  \item<5-> Schlussfolgerung
\end{itemize}
\end{frame}


\begin{frame}\frametitle{mehr zur Klasse Beamer}
\url{http://www.ctan.org/tex-archive/macros/latex/contrib/beamer/doc/beameruserguide.pdf}
\end{frame}