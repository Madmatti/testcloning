\documentclass[11pt]{article}
\usepackage{comment} %includecomment{Antwort} oder excludecomment{Antwort}
\usepackage[ngerman]{babel}\usepackage[babel=true]{microtype}
\usepackage[utf8]{inputenc}
\usepackage[T1]{fontenc}
\usepackage{lmodern}
\usepackage[a4paper,includeheadfoot,
	top=2cm, 
	bottom=2cm, 
	left=2cm, 
	right=2cm]{geometry}
\usepackage{color}
\usepackage{listings}
\usepackage[tikz]{mdframed}%\begin{mybox} \end{mybox}
\usepackage{amsmath,amssymb,amsfonts,amsthm}
\usepackage{graphicx} %\includegraphics[width=0.7\textwidth]{bild.jpg}
\usepackage{placeins} %for \FloatBarrier
\usepackage{caption} 
\usepackage[colorlinks=false,pdfborder={0 0 0}]{hyperref}%Hide RED PDF Border
\usepackage[showonlyrefs]{mathtools}%Math equations erstellen: \numberthis \label{key} abrufen:\ref{key}\eqref{key}
\usepackage{sectsty}
\usepackage{fancyhdr}\pagestyle{fancy}
\usepackage{xcolor} \definecolor{hintergrund}{HTML}{B0C4E3} \definecolor{cobalt}{rgb}{0.0, 0.28, 0.67}


\DeclareMathOperator*{\argmin}{argmin}
\DeclareMathOperator*{\argmax}{argmax}

\newcommand\numberthis{\addtocounter{equation}{1}\tag{\theequation}}
%%%%%%%%%%%%%%%%%%%%%%%%%%%%%%%%%%%%%%%%%%%%%%%%%%%%%%%%%%%%%%%%%%%
%                 Erstellt von Matthias Duch 2017                 %
%                                                                 %
% Programmiersprachen: (ausser den Standardsprachen)              %
% LaTeX, R, Stata                                                 %
%    \%%%%%%%%%%%%%%%%%%%%%%%%%%%%%%%%%%%%%%%%%%%%%%%%%%%%%%%%%%%%%%%%%%%
%                 Erstellt von Matthias Duch 2017                 %
%                                                                 %
% Programmiersprachen: (ausser den Standardsprachen)              %
% LaTeX, R, Stata                                                 %
%    \%%%%%%%%%%%%%%%%%%%%%%%%%%%%%%%%%%%%%%%%%%%%%%%%%%%%%%%%%%%%%%%%%%%
%                 Erstellt von Matthias Duch 2017                 %
%                                                                 %
% Programmiersprachen: (ausser den Standardsprachen)              %
% LaTeX, R, Stata                                                 %
%    \\input{../1.Vorlesung/Folien/settingsListing}               %
%%%%%%%%%%%%%%%%%%%%%%%%%%%%%%%%%%%%%%%%%%%%%%%%%%%%%%%%%%%%%%%%%%%
\usepackage{listings}
\definecolor{yellow}{RGB}{181,137,0}
\definecolor{red}{RGB}{220,50,47}
\definecolor{violet}{RGB}{108,113,196}
\definecolor{green}{rgb}{0,0.6,0}
\definecolor{gray}{rgb}{0.5,0.5,0.5}

\definecolor{Rblue}{HTML}{0000FF}
\definecolor{myblue}{HTML}{0000FF}




\newcommand{\colcomment}{\color{green}}
\newcommand{\colkeyword}{\color{blue}}
\newcommand{\coltexelem}{\color{green}}
\newcommand{\colnumbers}{\color{gray}}
\newcommand{\colstrings}{\color{violet}}
\newcommand{\colbackgrd}{\color{white}}

\lstset{literate= % Sonderzeichen ersetzen
  {á}{{\'a}}1 {é}{{\'e}}1 {í}{{\'i}}1 {ó}{{\'o}}1 {ú}{{\'u}}1
  {Á}{{\'A}}1 {É}{{\'E}}1 {Í}{{\'I}}1 {Ó}{{\'O}}1 {Ú}{{\'U}}1
  {à}{{\`a}}1 {è}{{\`e}}1 {ì}{{\`i}}1 {ò}{{\`o}}1 {ù}{{\`u}}1
  {À}{{\`A}}1 {È}{{\'E}}1 {Ì}{{\`I}}1 {Ò}{{\`O}}1 {Ù}{{\`U}}1
  {ä}{{\"a}}1 {ë}{{\"e}}1 {ï}{{\"i}}1 {ö}{{\"o}}1 {ü}{{\"u}}1
  {Ä}{{\"A}}1 {Ë}{{\"E}}1 {Ï}{{\"I}}1 {Ö}{{\"O}}1 {Ü}{{\"U}}1
  {â}{{\^a}}1 {ê}{{\^e}}1 {î}{{\^i}}1 {ô}{{\^o}}1 {û}{{\^u}}1
  {Â}{{\^A}}1 {Ê}{{\^E}}1 {Î}{{\^I}}1 {Ô}{{\^O}}1 {Û}{{\^U}}1
  {œ}{{\oe}}1 {Œ}{{\OE}}1 {æ}{{\ae}}1 {Æ}{{\AE}}1 {ß}{{\ss}}1
  {ű}{{\H{u}}}1 {Ű}{{\H{U}}}1 {ő}{{\H{o}}}1 {Ő}{{\H{O}}}1
  {ç}{{\c c}}1 {Ç}{{\c C}}1 {ø}{{\o}}1 {å}{{\r a}}1 {Å}{{\r A}}1
  {€}{{\euro}}1 {£}{{\pounds}}1 {«}{{\guillemotleft}}1
  {»}{{\guillemotright}}1 {ñ}{{\~n}}1 {Ñ}{{\~N}}1 {¿}{{?`}}1
}

\usepackage{accsupp}    
\lstset {
    numberstyle=\tiny\noncopynumber,
}

\newcommand{\noncopynumber}[1]{
    \BeginAccSupp{method=escape,ActualText={}}
    #1
    \EndAccSupp{}
}

% globale Einstellungen
\lstset{ %
  backgroundcolor=\colbackgrd,   % choose the background color; 
  basicstyle=\footnotesize,        % the size of the fonts that are used for the code
  breakatwhitespace=false,         % sets if automatic breaks should only happen at whitespace
  breaklines=true,                 % sets automatic line breaking
  emptylines=1, 
  captionpos=b,                    % sets the caption-position to bottom
  commentstyle=\itshape\colcomment,    % comment style
  deletekeywords={...},            % if you want to delete keywords from the given language
 % escapeinside={\%*}{*)},          % if you want to add LaTeX within your code
  extendedchars=true,              % lets you use non-ASCII characters; for 8-bits encodings only, does not work with UTF-8
  frame=single,	                   % adds a frame around the code
  keepspaces=true,                 % keeps spaces in text, useful for keeping indentation of code (possibly needs columns=flexible)
  keywordstyle=\colkeyword,       % keyword style
  otherkeywords={*,...},           % if you want to add more keywords to the set
  numbers=left,                    % where to put the line-numbers; possible values are (none, left, right)
  numbersep=1pt,                   % how far the line-numbers are from the code
  numberstyle=\sffamily\tiny\color{red!90}\noncopynumber, % the style that is used for the line-numbers
  numberfirstline=true,
  rulecolor=\color{black!70},         % if not set, the frame-color may be changed on line-breaks within not-black text (e.g. comments (green here))
  showspaces=false,                % show spaces everywhere adding particular underscores; it overrides 'showstringspaces'
  showstringspaces=false,          % underline spaces within strings only
  showtabs=false,                  % show tabs within strings adding particular underscores  showtabs=false,                 , % show tabs within strings adding particular underscores
  stepnumber=1,                    % the step between two line-numbers. If it's 1, each line will be numbered
  stringstyle=\colstrings,     % string literal style
  tabsize=2,	                   % sets default tabsize to 2 spaces
  title=\lstname,                   % show the filename of files included with \lstinputlisting; also try caption instead of title
  xleftmargin=3.4pt, % Einschub, durch die Zeilennummerierung bedingt
  xrightmargin=3.4pt,
  %
  alsoletter={.},
  %
  basicstyle=\ttfamily\color{black}, 
  columns=flexible,
}

%%%%%%%%%%%%%%%%%%%%%%%%%%%%%%%%%%%%%%%%%%%%%%%%%%%%%%%%%%%%%%%%%%%
%                           Sprachdefinitionen                    %
%%%%%%%%%%%%%%%%%%%%%%%%%%%%%%%%%%%%%%%%%%%%%%%%%%%%%%%%%%%%%%%%%%%

\lstdefinelanguage{Stata}{
    morekeywords=,
    morekeywords=[2]{cntrade, chinafin, wbopendata, spmap,},
    morekeywords=[3]{regress, summarize, display,},
    morekeywords=[4]{forvalues, if, foreach, set},
    morekeywords=[5]{rnormal, runiform},
    morecomment=[l]{//},
    morecomment=[s]{/*}{*/},
    % The following is used by macros, like `lags'.
    morecomment=[n][keywordstyle]{`}{'},
    morestring=[b]",
    sensitive=true,
}

\lstdefinestyle{Stata}{
  language=Stata,
  keywordstyle=\color{black},
  moredelim=**[is][\color{red}]{@}{@},
}

\lstdefinestyle{R}{%eigentlich nur R, aber aus Kompatibilitat zu alten Versionen
  style=base
}
\newcommand{\CodeSymbol}[1]{\textcolor{red}{#1}}

\lstdefinestyle{base}{
  language=R,
  emptylines=1,
  breaklines=true,
  alsoletter={.},
  basicstyle=\ttfamily\color{black},
  keywordstyle=\color{black},
  moredelim=**[is][\color{red}]{@}{@},
  %otherkeywords={<-,!,!=,~,$,*,\&,+,^,\%,\%/\%,\%*\%,\%\%,<<-,_,/},
%  morekeywords=,
%  morekeywords=[2]{+, chinafin, wbopendata, spmap,},
}

\lstdefinestyle{Latex}{
  language={[LaTeX]TeX},
  %
  moretexcs={geometry, definecolor, textcolor, colorbox, color, pagecolor, fcolorbox, setlength, tableofcontents, subsection, subsubsection, thepage, includegraphics, listoftables, citet, citep, eqref, fnysmbol, nameref, listoffigures, listoftables, url, href, printindex, blindtext, pause, titlepage, frametitle, subtitle, titlepage, usetheme,}, % define Commands
  morekeywords={maketitle}, %Keywords
  %
  texcsstyle=*\coltexelem, % Commandstyle
  moredelim=**[s][\color{yellow}]{[}{]},
}

%Patch issue, that closing parenthesis is not colored, if breaklines=True


               %
%%%%%%%%%%%%%%%%%%%%%%%%%%%%%%%%%%%%%%%%%%%%%%%%%%%%%%%%%%%%%%%%%%%
\usepackage{listings}
\definecolor{yellow}{RGB}{181,137,0}
\definecolor{red}{RGB}{220,50,47}
\definecolor{violet}{RGB}{108,113,196}
\definecolor{green}{rgb}{0,0.6,0}
\definecolor{gray}{rgb}{0.5,0.5,0.5}

\definecolor{Rblue}{HTML}{0000FF}
\definecolor{myblue}{HTML}{0000FF}




\newcommand{\colcomment}{\color{green}}
\newcommand{\colkeyword}{\color{blue}}
\newcommand{\coltexelem}{\color{green}}
\newcommand{\colnumbers}{\color{gray}}
\newcommand{\colstrings}{\color{violet}}
\newcommand{\colbackgrd}{\color{white}}

\lstset{literate= % Sonderzeichen ersetzen
  {á}{{\'a}}1 {é}{{\'e}}1 {í}{{\'i}}1 {ó}{{\'o}}1 {ú}{{\'u}}1
  {Á}{{\'A}}1 {É}{{\'E}}1 {Í}{{\'I}}1 {Ó}{{\'O}}1 {Ú}{{\'U}}1
  {à}{{\`a}}1 {è}{{\`e}}1 {ì}{{\`i}}1 {ò}{{\`o}}1 {ù}{{\`u}}1
  {À}{{\`A}}1 {È}{{\'E}}1 {Ì}{{\`I}}1 {Ò}{{\`O}}1 {Ù}{{\`U}}1
  {ä}{{\"a}}1 {ë}{{\"e}}1 {ï}{{\"i}}1 {ö}{{\"o}}1 {ü}{{\"u}}1
  {Ä}{{\"A}}1 {Ë}{{\"E}}1 {Ï}{{\"I}}1 {Ö}{{\"O}}1 {Ü}{{\"U}}1
  {â}{{\^a}}1 {ê}{{\^e}}1 {î}{{\^i}}1 {ô}{{\^o}}1 {û}{{\^u}}1
  {Â}{{\^A}}1 {Ê}{{\^E}}1 {Î}{{\^I}}1 {Ô}{{\^O}}1 {Û}{{\^U}}1
  {œ}{{\oe}}1 {Œ}{{\OE}}1 {æ}{{\ae}}1 {Æ}{{\AE}}1 {ß}{{\ss}}1
  {ű}{{\H{u}}}1 {Ű}{{\H{U}}}1 {ő}{{\H{o}}}1 {Ő}{{\H{O}}}1
  {ç}{{\c c}}1 {Ç}{{\c C}}1 {ø}{{\o}}1 {å}{{\r a}}1 {Å}{{\r A}}1
  {€}{{\euro}}1 {£}{{\pounds}}1 {«}{{\guillemotleft}}1
  {»}{{\guillemotright}}1 {ñ}{{\~n}}1 {Ñ}{{\~N}}1 {¿}{{?`}}1
}

\usepackage{accsupp}    
\lstset {
    numberstyle=\tiny\noncopynumber,
}

\newcommand{\noncopynumber}[1]{
    \BeginAccSupp{method=escape,ActualText={}}
    #1
    \EndAccSupp{}
}

% globale Einstellungen
\lstset{ %
  backgroundcolor=\colbackgrd,   % choose the background color; 
  basicstyle=\footnotesize,        % the size of the fonts that are used for the code
  breakatwhitespace=false,         % sets if automatic breaks should only happen at whitespace
  breaklines=true,                 % sets automatic line breaking
  emptylines=1, 
  captionpos=b,                    % sets the caption-position to bottom
  commentstyle=\itshape\colcomment,    % comment style
  deletekeywords={...},            % if you want to delete keywords from the given language
 % escapeinside={\%*}{*)},          % if you want to add LaTeX within your code
  extendedchars=true,              % lets you use non-ASCII characters; for 8-bits encodings only, does not work with UTF-8
  frame=single,	                   % adds a frame around the code
  keepspaces=true,                 % keeps spaces in text, useful for keeping indentation of code (possibly needs columns=flexible)
  keywordstyle=\colkeyword,       % keyword style
  otherkeywords={*,...},           % if you want to add more keywords to the set
  numbers=left,                    % where to put the line-numbers; possible values are (none, left, right)
  numbersep=1pt,                   % how far the line-numbers are from the code
  numberstyle=\sffamily\tiny\color{red!90}\noncopynumber, % the style that is used for the line-numbers
  numberfirstline=true,
  rulecolor=\color{black!70},         % if not set, the frame-color may be changed on line-breaks within not-black text (e.g. comments (green here))
  showspaces=false,                % show spaces everywhere adding particular underscores; it overrides 'showstringspaces'
  showstringspaces=false,          % underline spaces within strings only
  showtabs=false,                  % show tabs within strings adding particular underscores  showtabs=false,                 , % show tabs within strings adding particular underscores
  stepnumber=1,                    % the step between two line-numbers. If it's 1, each line will be numbered
  stringstyle=\colstrings,     % string literal style
  tabsize=2,	                   % sets default tabsize to 2 spaces
  title=\lstname,                   % show the filename of files included with \lstinputlisting; also try caption instead of title
  xleftmargin=3.4pt, % Einschub, durch die Zeilennummerierung bedingt
  xrightmargin=3.4pt,
  %
  alsoletter={.},
  %
  basicstyle=\ttfamily\color{black}, 
  columns=flexible,
}

%%%%%%%%%%%%%%%%%%%%%%%%%%%%%%%%%%%%%%%%%%%%%%%%%%%%%%%%%%%%%%%%%%%
%                           Sprachdefinitionen                    %
%%%%%%%%%%%%%%%%%%%%%%%%%%%%%%%%%%%%%%%%%%%%%%%%%%%%%%%%%%%%%%%%%%%

\lstdefinelanguage{Stata}{
    morekeywords=,
    morekeywords=[2]{cntrade, chinafin, wbopendata, spmap,},
    morekeywords=[3]{regress, summarize, display,},
    morekeywords=[4]{forvalues, if, foreach, set},
    morekeywords=[5]{rnormal, runiform},
    morecomment=[l]{//},
    morecomment=[s]{/*}{*/},
    % The following is used by macros, like `lags'.
    morecomment=[n][keywordstyle]{`}{'},
    morestring=[b]",
    sensitive=true,
}

\lstdefinestyle{Stata}{
  language=Stata,
  keywordstyle=\color{black},
  moredelim=**[is][\color{red}]{@}{@},
}

\lstdefinestyle{R}{%eigentlich nur R, aber aus Kompatibilitat zu alten Versionen
  style=base
}
\newcommand{\CodeSymbol}[1]{\textcolor{red}{#1}}

\lstdefinestyle{base}{
  language=R,
  emptylines=1,
  breaklines=true,
  alsoletter={.},
  basicstyle=\ttfamily\color{black},
  keywordstyle=\color{black},
  moredelim=**[is][\color{red}]{@}{@},
  %otherkeywords={<-,!,!=,~,$,*,\&,+,^,\%,\%/\%,\%*\%,\%\%,<<-,_,/},
%  morekeywords=,
%  morekeywords=[2]{+, chinafin, wbopendata, spmap,},
}

\lstdefinestyle{Latex}{
  language={[LaTeX]TeX},
  %
  moretexcs={geometry, definecolor, textcolor, colorbox, color, pagecolor, fcolorbox, setlength, tableofcontents, subsection, subsubsection, thepage, includegraphics, listoftables, citet, citep, eqref, fnysmbol, nameref, listoffigures, listoftables, url, href, printindex, blindtext, pause, titlepage, frametitle, subtitle, titlepage, usetheme,}, % define Commands
  morekeywords={maketitle}, %Keywords
  %
  texcsstyle=*\coltexelem, % Commandstyle
  moredelim=**[s][\color{yellow}]{[}{]},
}

%Patch issue, that closing parenthesis is not colored, if breaklines=True


               %
%%%%%%%%%%%%%%%%%%%%%%%%%%%%%%%%%%%%%%%%%%%%%%%%%%%%%%%%%%%%%%%%%%%
\usepackage{listings}
\definecolor{yellow}{RGB}{181,137,0}
\definecolor{red}{RGB}{220,50,47}
\definecolor{violet}{RGB}{108,113,196}
\definecolor{green}{rgb}{0,0.6,0}
\definecolor{gray}{rgb}{0.5,0.5,0.5}

\definecolor{Rblue}{HTML}{0000FF}
\definecolor{myblue}{HTML}{0000FF}




\newcommand{\colcomment}{\color{green}}
\newcommand{\colkeyword}{\color{blue}}
\newcommand{\coltexelem}{\color{green}}
\newcommand{\colnumbers}{\color{gray}}
\newcommand{\colstrings}{\color{violet}}
\newcommand{\colbackgrd}{\color{white}}

\lstset{literate= % Sonderzeichen ersetzen
  {á}{{\'a}}1 {é}{{\'e}}1 {í}{{\'i}}1 {ó}{{\'o}}1 {ú}{{\'u}}1
  {Á}{{\'A}}1 {É}{{\'E}}1 {Í}{{\'I}}1 {Ó}{{\'O}}1 {Ú}{{\'U}}1
  {à}{{\`a}}1 {è}{{\`e}}1 {ì}{{\`i}}1 {ò}{{\`o}}1 {ù}{{\`u}}1
  {À}{{\`A}}1 {È}{{\'E}}1 {Ì}{{\`I}}1 {Ò}{{\`O}}1 {Ù}{{\`U}}1
  {ä}{{\"a}}1 {ë}{{\"e}}1 {ï}{{\"i}}1 {ö}{{\"o}}1 {ü}{{\"u}}1
  {Ä}{{\"A}}1 {Ë}{{\"E}}1 {Ï}{{\"I}}1 {Ö}{{\"O}}1 {Ü}{{\"U}}1
  {â}{{\^a}}1 {ê}{{\^e}}1 {î}{{\^i}}1 {ô}{{\^o}}1 {û}{{\^u}}1
  {Â}{{\^A}}1 {Ê}{{\^E}}1 {Î}{{\^I}}1 {Ô}{{\^O}}1 {Û}{{\^U}}1
  {œ}{{\oe}}1 {Œ}{{\OE}}1 {æ}{{\ae}}1 {Æ}{{\AE}}1 {ß}{{\ss}}1
  {ű}{{\H{u}}}1 {Ű}{{\H{U}}}1 {ő}{{\H{o}}}1 {Ő}{{\H{O}}}1
  {ç}{{\c c}}1 {Ç}{{\c C}}1 {ø}{{\o}}1 {å}{{\r a}}1 {Å}{{\r A}}1
  {€}{{\euro}}1 {£}{{\pounds}}1 {«}{{\guillemotleft}}1
  {»}{{\guillemotright}}1 {ñ}{{\~n}}1 {Ñ}{{\~N}}1 {¿}{{?`}}1
}

\usepackage{accsupp}    
\lstset {
    numberstyle=\tiny\noncopynumber,
}

\newcommand{\noncopynumber}[1]{
    \BeginAccSupp{method=escape,ActualText={}}
    #1
    \EndAccSupp{}
}

% globale Einstellungen
\lstset{ %
  backgroundcolor=\colbackgrd,   % choose the background color; 
  basicstyle=\footnotesize,        % the size of the fonts that are used for the code
  breakatwhitespace=false,         % sets if automatic breaks should only happen at whitespace
  breaklines=true,                 % sets automatic line breaking
  emptylines=1, 
  captionpos=b,                    % sets the caption-position to bottom
  commentstyle=\itshape\colcomment,    % comment style
  deletekeywords={...},            % if you want to delete keywords from the given language
 % escapeinside={\%*}{*)},          % if you want to add LaTeX within your code
  extendedchars=true,              % lets you use non-ASCII characters; for 8-bits encodings only, does not work with UTF-8
  frame=single,	                   % adds a frame around the code
  keepspaces=true,                 % keeps spaces in text, useful for keeping indentation of code (possibly needs columns=flexible)
  keywordstyle=\colkeyword,       % keyword style
  otherkeywords={*,...},           % if you want to add more keywords to the set
  numbers=left,                    % where to put the line-numbers; possible values are (none, left, right)
  numbersep=1pt,                   % how far the line-numbers are from the code
  numberstyle=\sffamily\tiny\color{red!90}\noncopynumber, % the style that is used for the line-numbers
  numberfirstline=true,
  rulecolor=\color{black!70},         % if not set, the frame-color may be changed on line-breaks within not-black text (e.g. comments (green here))
  showspaces=false,                % show spaces everywhere adding particular underscores; it overrides 'showstringspaces'
  showstringspaces=false,          % underline spaces within strings only
  showtabs=false,                  % show tabs within strings adding particular underscores  showtabs=false,                 , % show tabs within strings adding particular underscores
  stepnumber=1,                    % the step between two line-numbers. If it's 1, each line will be numbered
  stringstyle=\colstrings,     % string literal style
  tabsize=2,	                   % sets default tabsize to 2 spaces
  title=\lstname,                   % show the filename of files included with \lstinputlisting; also try caption instead of title
  xleftmargin=3.4pt, % Einschub, durch die Zeilennummerierung bedingt
  xrightmargin=3.4pt,
  %
  alsoletter={.},
  %
  basicstyle=\ttfamily\color{black}, 
  columns=flexible,
}

%%%%%%%%%%%%%%%%%%%%%%%%%%%%%%%%%%%%%%%%%%%%%%%%%%%%%%%%%%%%%%%%%%%
%                           Sprachdefinitionen                    %
%%%%%%%%%%%%%%%%%%%%%%%%%%%%%%%%%%%%%%%%%%%%%%%%%%%%%%%%%%%%%%%%%%%

\lstdefinelanguage{Stata}{
    morekeywords=,
    morekeywords=[2]{cntrade, chinafin, wbopendata, spmap,},
    morekeywords=[3]{regress, summarize, display,},
    morekeywords=[4]{forvalues, if, foreach, set},
    morekeywords=[5]{rnormal, runiform},
    morecomment=[l]{//},
    morecomment=[s]{/*}{*/},
    % The following is used by macros, like `lags'.
    morecomment=[n][keywordstyle]{`}{'},
    morestring=[b]",
    sensitive=true,
}

\lstdefinestyle{Stata}{
  language=Stata,
  keywordstyle=\color{black},
  moredelim=**[is][\color{red}]{@}{@},
}

\lstdefinestyle{R}{%eigentlich nur R, aber aus Kompatibilitat zu alten Versionen
  style=base
}
\newcommand{\CodeSymbol}[1]{\textcolor{red}{#1}}

\lstdefinestyle{base}{
  language=R,
  emptylines=1,
  breaklines=true,
  alsoletter={.},
  basicstyle=\ttfamily\color{black},
  keywordstyle=\color{black},
  moredelim=**[is][\color{red}]{@}{@},
  %otherkeywords={<-,!,!=,~,$,*,\&,+,^,\%,\%/\%,\%*\%,\%\%,<<-,_,/},
%  morekeywords=,
%  morekeywords=[2]{+, chinafin, wbopendata, spmap,},
}

\lstdefinestyle{Latex}{
  language={[LaTeX]TeX},
  %
  moretexcs={geometry, definecolor, textcolor, colorbox, color, pagecolor, fcolorbox, setlength, tableofcontents, subsection, subsubsection, thepage, includegraphics, listoftables, citet, citep, eqref, fnysmbol, nameref, listoffigures, listoftables, url, href, printindex, blindtext, pause, titlepage, frametitle, subtitle, titlepage, usetheme,}, % define Commands
  morekeywords={maketitle}, %Keywords
  %
  texcsstyle=*\coltexelem, % Commandstyle
  moredelim=**[s][\color{yellow}]{[}{]},
}

%Patch issue, that closing parenthesis is not colored, if breaklines=True



%Seite Code einbinden:
%\lstinputlisting[language=R,style=base]{Tag1-Uebungsblatt.R}

\addto\captionsngerman{ %Abkürzen der Figureüberschriften
\renewcommand{\figurename}{Abb.}
\renewcommand{\tablename}{Tab.}
}
\lhead{Einführung in \LaTeX}
\chead{SoSe 2017} 
\rhead{Universität Bonn}
\lfoot{} \cfoot{\thepage}% \rfoot{Stand: \today}

\newmdenv[
  outerlinewidth      = 1,
  backgroundcolor     = red!25,
  roundcorner         = 5pt,
  frametitlebelowskip = 0pt,
  skipabove           = 0.5\baselineskip,
  skipbelow           = 0.1\baselineskip,
]{ansbox}

%\renewcommand{\labelenumi}{\alph{enumi})} % nummerieren mit Buchstaben
\newcommand{\code}[1]{\texttt{#1}} % Codeauszeichnung
\newcommand{\qcode}[1]{\glqq \texttt{\textbf{#1}}\grqq}
\newcommand{\bcode}[1]{\texttt{\textbf{#1}}} %fetter Code

\excludecomment{Antwort} %hier stehen lassen! Änderung im eig Dokument

% Überschrift Aufgabenblatt
\usepackage{calc}
\newcommand{\ueberschrift}[1]{\noindent \colorbox{hintergrund}{\parbox[t]{\textwidth - 2 \fboxsep}{\centering \LARGE \textbf{#1:}}}}
\usepackage{titlesec}
\sectionfont{\color{cobalt}} \subsectionfont{\color{cobalt}}
\titleformat{\section}
 {}{}{1em}{{\color{cobalt}\normalfont\bfseries Aufgabe \thesection.}~}
\usepackage{framed}
\newcommand\result[1]{\vspace{\abstandVorResult}\begin{framed}#1\end{framed}}
